\documentclass[12pt,a4paper]{article}
\usepackage[a4paper,left=2cm,right=2cm, top=2cm, bottom=3cm]{geometry}
\usepackage[utf8]{inputenc}
\usepackage[brazil]{babel}
\usepackage[T1]{fontenc}
\usepackage{amsmath}
\usepackage{amsfonts}
\usepackage{amssymb}
\usepackage{graphicx}
\usepackage{amsthm,amsfonts,amstext}
\usepackage{enumerate}
\usepackage{xcolor}
\usepackage{tikz}
\usetikzlibrary{arrows}
\usetikzlibrary{automata,arrows,positioning}
\tikzset{
 node distance=2.5cm,
 initial text={$M_{indice}$},
 double distance=1pt,
 every state/.style={semithick,fill=blue!20!white,minimum size=20pt,inner sep=0pt},
 every edge/.style={draw,->,>=stealth,auto,semithick,font=\ttfamily\small}
}

\title{LateX - Basic Codes for Professional Texts}
\author{Richardson L. Bessa Filho
}
\date{ }

\usepackage{natbib}
\usepackage{graphicx}

\begin{document}

\maketitle

\section*{Title}

\section{Section}
%---------------------------------------------------------------
\subsection{Subsection}
%---------------------------------------------------------------
\subsubsection{Subsubsection}
\begin{enumerate}
%---------------------------------------------------------------
    \item Enumeration example:
    
        \begin{enumerate}
            \item First item
            \item Second item
        \end{enumerate}
%---------------------------------------------------------------
\item Itemize (*) example:

        \begin{itemize}
            \item First item
            \item Second item
        \end{itemize}
%---------------------------------------------------------------
\item Itemize (I) example:
        
        need to include \verb|\usepackage{enumerate}| 
        \begin{enumerate}[I]
            \item First item
            \item Second item
        \end{enumerate}
%---------------------------------------------------------------
\item Itemize (i) example:
        
        need to include \verb|\usepackage{enumerate}| 
        \begin{enumerate}[i]
            \item First item
            \item Second item
        \end{enumerate}
%---------------------------------------------------------------
\item Text color example:

      need to include \verb|\usepackage{xcolor}| 

\textcolor{blue}{BLUE}

\textcolor{red}{RED}

\textcolor{green}{GREEN}


%---------------------------------------------------------------
\item Example text in inline math mode (default):

You must enclose the text between \$\verb|expression|\$

$ax^2 + bx +c = 0$

$(A \cup B) \cap C = \{x \in \mathbb{N} \mid x^2 < 25\}$

$\neg q \rightarrow \neg p$

$\mathcal{L}_{A21}$: $G_2=(V,\Sigma,P,S)=(\{A,B,C,D,E,F,S\},\{0,1, \varepsilon \},P,S)$
%---------------------------------------------------------------
\item Example of text in math mode highlighted (centered):
You must enclose the text between \$\$\verb|expression|\$\$

$$ x=\frac{-b\pm\sqrt{b^2-4ac}}{2a} $$

$$\sum_{i=1}^{n} a_i$$

$$ \binom{n+1}{k} $$
%---------------------------------------------------------------
\item Figure inclusion

        need to include \verb|\usepackage{graphicx}| 

\begin{figure}[ht]
\centering
\includegraphics[scale=0.7]{MP1.jpg}
\caption{Rusty metal image}
\end{figure}
%---------------------------------------------------------------
\newpage
\item Table example 1:

\begin{table}[h]
\centering
\begin{tabular}{|c|c|c|c|c|c|c|}
\hline
 default & 1 & 0 & 0 & 1 & 0 & 1 \\ \hline
 \textbf{bold} & 1 & 0 & 0 & 1 & 0 & 1 \\
 \textit{italics} & 1 & 0 & 0 & 1 & 0 & 1 \\ 
  \textsc{capslock} & 1 & 0 & 0 & 1 & 0 & 1 \\
  \textsuperscript{superscript} & 1 & 0 & 0 & 1 & 0 & 1 \\
  \verb|reserved| & \% & \# & \{ & \} & 0 & 1 \\
\hline
\end{tabular}
\end{table}
%---------------------------------------------------------------

\item Table example 2:
  $$
  \begin{array}{ccll}
  \hline
  \textbf{State} & \textbf{ID} & \textbf{Expression} & \textbf{Action}\\
  \hline
    I & 1 & S = 0A \cup 1B           &\\
      & 2 & A = 0B \cup 1B \cup \varepsilon          &\\
      & 3 & B = 0A \cup 1A           &\\
  \hline
   II & 1 & S = 0A \cup 1S                 &\\
      & 2 & A = 0^*1B                      & I.2 \to \texttt{Lema de Arden}\\
      & 3 & B = 0C \cup 1S                 &\\
      & 4 & C = (0\cup 11)C \cup 1\cup \varepsilon & I.5 \to I.4, \text{Fatoração}\\
  \hline
  III & 1 & S = 0A \cup 1S              &\\
      & 2 & A = 0^*10C\cup 0^*11S       & II.3 \to II.2\\
      & 3 & C = (0\cup 11)^*(1\cup \varepsilon) & \text{Lema de Arden}\\
  \hline
   IV & 1 & S = 0A \cup 1S                              &\\
      & 2 & A = 0^*10(0\cup 11)^*(1\cup \varepsilon)\cup 0^*11S & III.3 \to III.2\\
  \hline
    V & 1 & S = 0^+10(0\cup 11)^*(1\cup \varepsilon)\cup 0^+11S \cup 1S & IV.2 \to IV.1\\
  \hline
   VI & 1 & S = (1\cup 0^+11)^*0^+10(0\cup 11)^*(1\cup \varepsilon) & \text{Fatoração, Lema de Arden}\\ 
   \hline
  \end{array}
  $$
  
  %-------------------------------------------------------------
\item  Array example:
  $$ 
   P =
   \left\{\begin{array}{l}
    S \to A \mid D,\\
    A \to 0B,\\
    B \to 0C \mid 1C \mid \varepsilon \\
    C \to 0B \mid 1B,\\
    D \to 1E\\
    E \to 0F \mid 1F,\\
    F \to 0E \mid 1E \mid \varepsilon
   \end{array}\right\}.
  $$
%---------------------------------------------------------------
 \item Derivation tree example:
 
        need to include \verb|\usepackage{tikz}|
        
        need to include \verb|\usetikzlibrary{arrows}|
\begin{center}
    % size
   \begin{tikzpicture}[
    ->,>=stealth',
    level distance=1.3cm,
    level 2/.style={sibling distance=2.0cm},
    level 3/.style={sibling distance=1cm},
    transform shape,
    scale=.8
   ]
   % tree construction
    \node (t1) {$S$} %start node
    child { 
        node {0} 
    }
    child { 
        node {$S$}
        child { 
            node {$A$}
            child { 
                node {0} 
            }
            child { 
                node {$A$} 
                child { 
                    node {$\varepsilon$}
                }
            }
            child { 
                node {$1$} 
            } 
        }
    }
    child { node {0} };
%      
   
   \end{tikzpicture}
  \end{center}
%-----------------------------------------------------------------------------

\item Automata:

        need to include \verb|\usetikzlibrary{arrows}|
        
        need to include \verb|\usetikzlibrary{automata,arrows,positioning}|
        
        % automata size:

\tikzset{
 node distance=2.5cm, % assigns the distance between nodes for all graphs
 initial text={$M_{id}$}, % start point text
 double distance=1pt,
 every state/.style={semithick,fill=blue!20!white,minimum size=20pt,inner sep=0pt},
 every edge/.style={draw,->,>=stealth,auto,semithick,font=\ttfamily\small}
}
        
        \begin{center} %centralize
 \begin{tikzpicture}[
  node distance=2.cm, % assigns the distance between nodes only to the following graph
   transform shape,
   scale=1.1
  ]
  %add nodes and assign them states and conditions
   \node[state,initial]                                   (s0) {$s_0$};
   \node[state,right of=s0]                               (s1) {$s_1$};
   \node[state,right of=s1]                               (s2) {$s_2$};
   \node[state,fill=green!20!white,accepting,right of=s2] (s3) {$s_3$};
   \node[state,fill=green!20!white,accepting,below of=s3] (s4) {$s_4$};
   \node[state,fill=red!20!white,left of=s4]              (s5) {$s_5$};
 %add edges and their weights/behaviours
   \draw (s0) edge               node {22}   (s1)
              edge[loop below]   node {13}   (s0)
         (s1) edge[loop above]   node {7}   (s1)
              edge               node {19}   (s2)
         (s2) edge               node {42}   (s3)
              edge[bend left=30] node {101}   (s0)
         (s3) edge[loop right]   node {00000}   (s3)
              edge[bend left=20] node {11111}   (s4)
         (s4) edge               node {0212121}   (s5)
              edge[bend left=20] node {1}   (s3)
         (s5) edge[loop left]    node {0,1} (s5)
         ; %insert ; after defining the edges to indicate the end of "\draw"
 \end{tikzpicture}
\end{center}
\end{enumerate}

\end{document}
